\documentclass[a4paper, 12pt]{book}
\usepackage[T2A]{fontenc}
\usepackage[utf8]{inputenc}
\usepackage[english,russian]{babel}
\usepackage{amsmath, amsfonts, amssymb, amsthm, mathtools, misccorr, indentfirst, multirow}
\usepackage{wrapfig}
\usepackage{graphicx}
\usepackage{subfig}
\usepackage{adjustbox}

\title{Билеты по общей физике. 4-й семестр. Оптика.}

\begin{document}
	\maketitle
	\newpage
	\tableofcontents
	\chapter{Список вопросов}
	\begin{enumerate}
		\item Волновое уравнение. Монохроматические волны. Комплексная амплитуда. Уравнение Гельмгольца.
		\item Монохроматические волны. Комплексная амплитуда. Уравнение плоской и сферической волн. Принцип суперпозиции, интерференция.
		\item Интерференция монохроматических волн. Интерференция плоской и сферической волн. Ширина интерференционных полос. Видность полос.
		\item Влияние немонохроматичности света на видность интерференционных полос. Функция временной когерентности. Связь времени когерентности с шириной спектра. Теорема Винера-Хинчина. Соотношение неопределенностей.
		\item Видность интерференционных полос и ее связь со степенью когерентности при использовании квазимонохроматических источников света. Оценка максимального числа наблюдаемых полос. Максимально допустимая разность хода в интерференционных опытах.
		\item Апертура интерференционной схемы и влияние размеров источника на видность интерференционных полос. Функция пространственной когерентности. Радиус пространственной когерентности.
		\item Связь радиуса пространственной когерентности с угловым размером протяженного источника. Теорема Ван-Циттерта-Цернике. Видность интерференционных полос при использовании протяженных источников света. Звездный интерферометр Майкельсона.
		\item 
	\end{enumerate}
\end{document}
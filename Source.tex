\documentclass[a4paper, 12pt]{book}
\usepackage[T2A]{fontenc}
\usepackage[utf8]{inputenc}
\usepackage[english,russian]{babel}
\usepackage{amsmath, amsfonts, amssymb, amsthm, mathtools, misccorr, indentfirst, multirow}
\usepackage{wrapfig}
\usepackage{graphicx}
\usepackage{subfig}
\usepackage{adjustbox}

\title{Билеты по общей физике. 4-й семестр. Оптика.}

\begin{document}
	\maketitle
	\newpage
	\tableofcontents
	\chapter{Список вопросов}
	\begin{enumerate}
		\item Волновое уравнение. Монохроматические волны. Комплексная амплитуда. Уравнение Гельмгольца.
		\item Монохроматические волны. Комплексная амплитуда. Уравнение плоской и сферической волн. Принцип суперпозиции, интерференция.
		\item Интерференция монохроматических волн. Интерференция плоской и сферической волн. Ширина интерференционных полос. Видность полос.
		\item Влияние немонохроматичности света на видность интерференционных полос. Функция временной когерентности. Связь времени когерентности с шириной спектра. Теорема Винера-Хинчина. Соотношение неопределенностей.
		\item Видность интерференционных полос и ее связь со степенью когерентности при использовании квазимонохроматических источников света. Оценка максимального числа наблюдаемых полос. Максимально допустимая разность хода в интерференционных опытах.
		\item Апертура интерференционной схемы и влияние размеров источника на видность интерференционных полос. Функция пространственной когерентности. Радиус пространственной когерентности.
		\item Связь радиуса пространственной когерентности с угловым размером протяженного источника. Теорема Ван-Циттерта-Цернике. Видность интерференционных полос при использовании протяженных источников света. Звездный интерферометр Майкельсона.
	\end{enumerate}
	\chapter{Волновое уравнение.}
	\section{Волновое уравнение}
	Волной называется процесс, обладающий некоторым свойством инвариантности, что некоторая физическая величина представляется профилем перемещающимся с постоянной скоростью:
	\begin{equation}
		S\left(x,t\right)\equiv S\left(x\pm ut\right)
		\label{wave_speed}
	\end{equation}
	Волновое уравнение можно получить дважды продифференцировав (\ref{wave_speed}):
	\begin{equation}
		\frac{\partial^2 S}{\partial t^2}=\frac{u^2\partial^2 S}{\partial x^2}
	\end{equation}
	В общем трехмерном случае
	\begin{equation}
		\nabla^2 S-\frac{1}{u^2}\frac{\partial^2 S}{\partial t^2}=0
	\end{equation}
	\paragraph{Монохроматическая волна}— это строго синусоидальная волна с постоянной во времени \textbf{частотой} $\omega$, \textbf{амплитудой} $a$ и \textbf{начальной частотой} $\varphi$.\par
	В общем случае имеет вид
	\begin{equation}
		S\left(\vec{r}, t\right)=a\left(\vec{r}\right)\cos\left(\omega t-\varphi\left(\vec{r}\right)\right).
	\end{equation}
	Несколько важных примеров простейших типов монохроматических волн:
	\subsection{Плоская монохроматическая волна}
	Описывается функцией координат и времени вида:
	\begin{equation}
		S\left(z,t\right)=a\cos\left(\omega t-kz-\varphi\right)
		\label{flat_mono_wave_eq}
	\end{equation}
	$a$ — амплитуда волны, $\Phi=\omega t-kz-\varphi$ — фаза волны, $\varphi=\Phi\left(z=0,t=0\right)$ — начальная фаза.\par
	Из уравнения (\ref{flat_mono_wave_eq}) видно, что в плоскости $z=\mathrm{const}$ колебания происходят по одному и тому же закону с одной и той же частотой, амплитудой и одной и той же начальной фазой $\varphi$. Поверхности, на которых колебания возмущения $S$ происходят синфазно называются \textbf{волновыми поверхностями}.\par
	\subsection{Сферическая волна}
	Волна, описываемая уравнением
	\begin{equation}
		S\left(\vec{r},t\right)=\frac{a}{r}\cos\left(\omega t-kr-\varphi_0\right)
	\end{equation}
	называется \textbf{сферической}.
	\section{Комплексная амплитуда волны}
	В самом общем виде уравнение \textbf{монохроматической волны}
	\begin{equation}
		S\left(\vec{r},t\right)=a\left(\vec{r}\right)\cos\left(\omega t-\varphi\left(\vec{r}\right)\right)
		\label{general_wave_eq}
	\end{equation}
	Наряду с волной (\ref{general_wave_eq}) рассмотрим волновой процесс вида
	\begin{equation*}
		S_1\left(\vec{r},t\right)=a\left(\vec{r}\right)\sin\left(\omega t-\varphi\left(\vec{r}\right)\right)
	\end{equation*}
	Ясно, что линейная комбинация функций вида
	\begin{equation}
		V\left(\vec{r},t\right)=S\left(\vec{r},t\right)-iS_1\left(\vec{r},t\right)
		\label{general_linear_wave_eq}
	\end{equation}
	также удовлетворяет волновому уравнению.\par
	Используя тождество $\cos\alpha-i\sin\alpha=e^{-i\alpha}$, перепишем (\ref{general_linear_wave_eq}) в виде
	\begin{equation*}
		V\left(\vec{r},t\right)=a\left(\vec{r}\right)e^{-i\left[\omega t-\varphi\left(\vec{r}\right)\right]}
	\end{equation*}
	Получаем, что функцию $V\left(\vec{r},t\right)$ можно записать в виде произведения двух функций
	\begin{equation*}
		V\left(\vec{r},t\right)=f\left(\vec{r}\right)e^{-i\omega t}
	\end{equation*}
	где $f\left(\vec{r}\right)=a\left(\vec{r}\right)e^{i\varphi\left(\vec{r}\right)}$ есть \textbf{комлексная амплитуда волны}.
	\section{Уравнение Гельмгольца}
	Комплексная функция $V\left(\vec{r},t\right)=f\left(\vec{r}\right)e^{-i\omega t}$ должна быть решением волнового уравнения
	\begin{equation*}
		\nabla^2 V-\frac{1}{u^2}\frac{\partial^2 V}{\partial t^2}=0.
	\end{equation*}
	Дифференцируя $V\left(\vec{r},t\right)$ дважды по координатам получаем
	\begin{equation*}
		\nabla^2 V=e^{-i\omega t}\nabla^2 f
	\end{equation*}
	Дифференцируя дважды по времени
	\begin{equation*}
		\frac{\partial^2 V}{\partial t^2}=f\left(\vec{r}\right)\left(-i\omega\right)^2e^{-i\omega t}
	\end{equation*}
	подставляя выражение $\nabla^2 V$ и $\frac{\partial^2 V}{\partial t^2}$ в волновое уравнение приходим к следующему равенству
	\begin{equation}
		\nabla^2 f+k^2 f=0
	\end{equation}
	где $k=\frac{\omega}{v}$ волновое число. Полученное уравнение для комплексных амплитуд называется \textbf{уравнением Гельмгольца}.
	\chapter{Уравнения плоской и сферической волн.}
	\section{Интерференция плоских волн}
	Рассмотрим результат интерференции двух плоских волн, волновые векторы $\vec{k}_1$ и $\vec{k}_2$ которых составляют углы $\pm\alpha$ с нормалью к плоскости.
	\begin{equation*}
		f_1\left(x,z\right)=a_1e^{i\left(kx\sin\alpha+kz\cos\alpha\right)},
		\quad
		f_2\left(x,z\right)=a_2e^{i\left(-kx\sin\alpha+kz\cos\alpha\right)}
	\end{equation*}
	Результирующую картину интенсивности найдем, используя \textbf{общее соотношение}
	\begin{equation}
		I\left(x\right)=a_1^2+a_2^2+2a_1a_2\cos\left(2kx\sin\alpha\right)
	\end{equation}
	Картина имеет вид чередующихся светлых и темных полос. Ширина полос:
	\begin{equation}
		\Delta x=\frac{\lambda}{2\sin\left(\alpha\right)}
	\end{equation}
	\section{Интерференция сферических волн}
	Две сферические волны излучаются точечными источниками $S_1$ и $S_2$. Комплексные амплитуды волн в точке наблюдения есть
	\begin{equation*}
		f_1=\frac{a_0}{r_1}e^{ikr_1},\quad f_2=\frac{a_0}{r_2}e^{ikr_2}
	\end{equation*}
	Разность фаз в точке наблюдения $\Delta\varphi=k\cdot\Delta$, где $\Delta=r_2-r_1$ — разность хода волн, приходящих в точку.
	Если рассматривать небольшую область наблюдения, в которой амплитуды двух слагаемых волн примерно одинаковы: $a_0/r_1\approx a_0$, $a_0/r_2\approx a_0$, то получаем
	\begin{equation}
		I=2I_0\left[1+\cos\left(\frac{\omega}{c}\right)\Delta\right]
	\end{equation}
	\section{Уравнение плоской и сферической волн.}
	\begin{enumerate}
		\item Для \textbf{плоской}
		\begin{equation*}
			S\left(\vec{r},t\right)=a\cos\left(\omega t-\vec{k}\cdot\vec{r}-\varphi\right)
		\end{equation*}
		и в комплексной форме комплексная амплитуда для плоской волны имеет вид
		\begin{equation*}
			f\left(\vec{r}\right)=ae^{i\varphi}e^{i\vec{k}\cdot\vec{r}}=ce^{i\left(k_xx+k_yy+k_zz\right)}
		\end{equation*}
		\item Для \textbf{сферической волны}
		\begin{equation*}
			S\left(\vec{r},t\right)=\frac{a}{r}\cos\left(\omega t-kr-\varphi_0\right)
		\end{equation*}
		и в комплексной форме комплексная амплитуда для сферической волны имеет вид
		\begin{equation*}
			f\left(\vec{r}\right)=\frac{a}{r}e^{i\varphi_0}e^{ikr}=\frac{a_0}{r}e^{ikr+\varphi_0}
		\end{equation*}
	\end{enumerate}
\end{document}